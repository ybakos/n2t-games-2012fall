\section{Objective}
    In a quest to spread the joys of using a Reverse Polish Notation (RPN)
    Calculator, and the efficiency of the stack, a basic calculator will be
    presented. This calculator will do most everything a calculator should, it
    will add, subtract, multiply, and divide. The only difference to a regular
    calculator, will be in its strange, to most, operational syntax.

\section{Postfix vs. Infix}
    RPN style calculators have been around for a while and are definitely not a
    new concept, and were born out of simplicity. RPN does math using postfix
    notation, meaning input looks like \texttt{2 5 +}. Infix, what we think of
    when we think calculator these days, uses notation that is very similar
    (sometimes identical) to how we actually write math, \texttt{2 + 5}. Infix
    lends itself to readability, while postfix lends itself to efficiency.

\section{Goals}
    \begin{itemize}
        \item Basic REPL, for user interface
        \item A working stack
        \item 4 basic mathematical operations: + - * /
    \end{itemize}
    \subsection{Secondary Goals}
        \begin{itemize}
            \item Display of top items on stack on screen
            \item Additional math operations: sqrt and pow
            \item Stack manipulation: swap and dup
            \item (Really ambitious) User variables
        \end{itemize}

\section{Concerns}
    The biggest concern going into this project will be the limited ROM and RAM of
    the Jack machine. Additionally the limited standard library of the Jack
    language will prove to be quite challenging, since a lot of underlying data
    structures will have to be implemented.

\section{User Interface}
    The user interface will be a simple text based interface. The user will have
    a basic input line at bottom of line. Bad input checking will be minimal at
    best so the program is relying on the user to enter valid input.
    Additionally every element to the stack must be entered separately,
    simply separating elements with spaces will not be accepted. While this
    interface may sound limited to other similar programs (dc), it does not at all
    reduce the functionality.
